\documentclass[cn,hazy,blue,14pt,screen]{elegantnote}
% \documentclass[cn,hazy,blue,14pt,normal]{elegantnote}

% 导言区

\title{R语言数据预处理\textcolor{orange}{笔记}}

\author{缪啸宇}
\institute{南航经管}
\date{\zhtoday}

\usepackage{array}
\usepackage{listings}   % 这个包是用于美化代码块的
\usepackage{xcolor}
\usepackage{hyperref}

\lstset{  % 这个是代码块的格式设置
 columns=fixed,       
 numbers=left,                                        % 在左侧显示行号
 numberstyle=\tiny\color{gray},                       % 设定行号格式
 frame=none,                                          % 不显示背景边框
 backgroundcolor=\color[RGB]{245,245,244},            % 设定背景颜色
 keywordstyle=\color[RGB]{40,40,255},                 % 设定关键字颜色
 numberstyle=\footnotesize\color{darkgray},           
 commentstyle=\it\color[RGB]{0,96,96},                % 设置代码注释的格式
 stringstyle=\rmfamily\slshape\color[RGB]{128,0,0},   % 设置字符串格式
 showstringspaces=false,                              % 不显示字符串中的空格
 language=R,                                          % 设置语言
}

% ———————————————————————————
% 正文区
\begin{document}

\maketitle

\centerline{
  \includegraphics{lijie}}

% 第一章 
\newpage    % 不加这个命令的话,正文内容会出现的封面中。这个是让正文从新的一页开始的命令
\section{数据导入工具}
\noindent 需要掌握:
\begin{itemize}
    \item R语言如何读取不同类型的数据文件
    \item 读取数据常用的R包
    \item 不同R包中相似函数的优缺点
    \item 数据读取函数的常用参数设置
    \item 处理规则和不规则原始数据文件
\end{itemize}

% 1.1节
\subsection{utils——一个自带的基础包}
utils不是专门用来进行数据导入的。utils比较底层,是用来编程和开发R包的。导入数据用到这个包是为了学习时事半功倍。

% 1.1.1节
\subsubsection{read.csv/csv2——逗号分隔}
.csv = comma-separated values,即每个数据值用逗号隔开,因此本质上是一种.txt。而utils包中的\lstinline{read.csv/csv2}\footnote{read.csv是逗号分隔;read.csv2是分号分隔。一般用csv多,因为csv里小数点是“.”,而csv2里小数点是“,”。}正是用于快速读取csv文件。read.csv的简单实用方式如下\footnote{file参数可以省略,即read.csv("flights.csv")}:

\begin{lstlisting}
  flights <- read.csv(file = "flights.csv")
\end{lstlisting} 

要会使用\textcolor{red}{.rproj},这个是R项目,将每一次数据分析过程都看作一个独立的项目。其作用是免去设置路径的麻烦,也减少了原始数据文件太多导致的各种隐患。

数据导入,第一步通常是调用str函数来进行初步检视:

\begin{lstlisting}
  str(object = flights)
\end{lstlisting} 

str函数用于检视数据的结构、变量名称和类型等。除了str外,还有其他检视数据集的函数如head、tail、view等。检视数据集的意义在于.csv文件并不一定是逗号分隔的。例如read.csv读取Tab分隔的文件就会出现把所有数据读取到了一列中的情况。这在检视结果中可以看出。该问题可以用指定分隔符参数解决,将\textcolor{magenta}{$\backslash$t}指定给sep参数后就可以用read.csv读取Tab分隔的csv文件,如下:

\begin{lstlisting}
  flights3 <- read.csv(file = "flights1.csv", sep = "\t")
  str(flights3)
\end{lstlisting}

read.csv默认将所有字符型数据都读成因子型(Factor),这会对后续处理带来麻烦,因此将字符因子化关掉,用\textcolor{magenta}{stringsAsFactors}参数:

\begin{lstlisting}
  flights_str <- read.csv("flightsstrings.csv", sep = "\t", stringsAsFactors = FALSE)
  str(flights_str)
\end{lstlisting}

% 1.1.2节
\subsubsection{read.delim/delim2——tab分隔}
\lstinline{read.delim/delim2}\footnote{delim读取小数点是“.”的数据;delim2读取小数点是“,”的数据}是专门处理tab分隔的数据文件的。

其与read.csv的唯一不同只是参数\textcolor{magenta}{sep = "$\backslash$t"}。因此和前面差不多,read.delim会在原始数据中不断寻找tab分隔符,如果莫得的话就会和前面一样将所有变量都挤在一列中。实例代码如下,可以看出和read.csv的参数格式相同;还可以看出其本质是通过read.table执行的\footnote{read.csv/csv2/delim/delim2的母函数都是read.table}。

\begin{lstlisting}
  read.delim
\end{lstlisting}

% 1.1.3节
\subsubsection{read.table——任意分隔符}

\lstinline{read.table}函数将文件读取成数据框的格式,示例:

\begin{lstlisting}
  flights <- read.table("flights.csv", header = TRUE)
  head(x = flights)
\end{lstlisting}

上面head函数类似str函数,只不过head函数是从上到下显示,str函数是从左到右显示。一般推荐两个都运行下,head方便与原始数据比对,str显示所保存的数据框属性、变量类型等信息。

\begin{itemize}
  \item 如果不将header设置为TRUE的话,原数据里变量名称这一行就会变成观测值的第一行,变量名会被函数重新分配一个V1。
  \item 如果不将分隔符sep参数设置为“,”的话,由于read.table的默认分隔符是空白\footnote{空白≠空格。那等于啥啊草,书里也没说。},会导致所有变量挤在一列中。
  \item 不要忘记加上\textcolor{magenta}{stringsAsFactors = FALSE},让字符型变量不要变成因子型。
  \item 关于缺失值的处理感觉这个讲得不太好,需要另外查一下。
\end{itemize}

数据集里出现默认值(NA)或空白(“ ”)很常见。理论上,默认值仍是观测值,而空白可能不是数据。这个可以拓展的东西比较多,这边之后可以放一个超链接。

% 1.2节
\subsection{readr——进阶数据读取}
\textcolor{red}{readr}的优势:

\begin{itemize}
  \item 更快。
  \item 默认设置简洁。会自动解析每列数据类型并显示解析结果,无需设置stringAsFactors。
  \item 对数据类型的解析更准确\footnote{read.table在甄别数据属性时,只会对起始5行的观测值评估;readr是1000行。这在处理空白行时免去了很多麻烦。}。
\end{itemize}

readr中常用的数据读取函数:

\begin{itemize}
  \item \lstinline{read_delim}\footnote{read\_delim是read\_csv/csv2/tsv的母函数,因此可以直接调用子函数。其分别对应utils中的read.csv/csv2/delim。}
  \item read\_fwf
  \item read\_lines
  \item read\_log
  \item read\_table
\end{itemize}

\lstinline{read_delim}中比较重要的参数:

\begin{itemize}
  \item file:路径+文件名\footnote{.gz/bz2/xz/zip的会自动解压缩;http(s)://和ftp(s)://会下载到本地}。
  \item delim:每行的分隔符,如“,”和“$\backslash$t”。不过在三个子函数中就无需设置,只是母函数没有默认值而已。
  \item skip:跳过几行读取原始数据文件,默认为0,即不跳过。
  \item col\_names:TRUE时原始数据文件第一行用作列名,且不在数据集内。其他的自查。
  \item col\_types:列的数据类型,内容较多,自查。
\end{itemize}

read\_csv运行后会在console中显示解析后每一列的属性,可以快速甄别哪一列的属性和期望不符。

\begin{lstlisting}
  library(readr)
  read_csv("flights.csv")
\end{lstlisting}

% 1.3节
\subsection{utils vs readr}

\begin{itemize}
  \item 用utils导入数据:莫得网络,装不了包,不想多加载一个包。
  \item 用readr导入数据:不介意多加载一个包,喜欢用默认设置。
  \item 用data.table包中的fread函数:文件个数(>10)和文件大小(>1MB)较大时用。
\end{itemize}

% 1.4节
\subsection{readxl——Excel文件}

读取Excel文件的必备包,只有五个函数:

\begin{itemize}
  \item excel\_format和excel\_sheets:探测性函数
  \item readxl\_example:引用例子
  \item cell-specification:读取特定单元格
  \item \lstinline{read_excel}:最重要的,读取Excel文件
\end{itemize}

现在数据会被读取为tibble格式,可以理解为提升版的data.frame。

% 1.5节
\subsection{DBI——数据库查询下载}

\textcolor{magenta}{DBI包}能够方便地与数据库进行交互,需要有以下的前提:
\begin{itemize}
  \item 已知数据库类型(如MySQL、MongoDB等)
  \item 安装了相应数据库类型的R包
  \item 数据库服务器地址
  \item 数据库名称
  \item 接入数据库的权限、账号、密码
  \item 安装dplyr包用来本地化数据库中的数据
\end{itemize}

R与数据库交互的流程为:建立连接→发送查询请求→获取相关数据。需要三个必备的包\footnote{DBI和RPostgreSQL建立与数据库的连接和发送请求;dplyr将数据保存到本地}:

\begin{lstlisting}
  library(DBI)
  library(dplyr)
  library(RPostgreSQL)
\end{lstlisting}

\begin{itemize}
  \item dbConnect:连接服务器,需要在单引号中输入服务器地址、名称、权限等信息。
  \item dbListTables:查询数据库中的详细内容,用字符串向量的格式返回。无内容则返回空值。
\end{itemize}

这块的内容这里只是简单介绍,数据库交互详细可以查阅其他资料。

% 1.6节
\subsection{pdftools——PDF文件}
一般很少遇到PDF文件,这些大多是学术期刊、杂志、电子书籍等。不过在文本挖掘(Text Mining)和主题模型(Topic Modelling)预测中,pdftools包必备。该包只有两个母函数:从PDF中提取数据(pdf\_info);将文件渲染成PDF格式。

\textcolor{magenta}{pdf\_info}母函数下一共有6个子函数:

\begin{itemize}
  \item pdf\_info:读取PDF文件的基本信息。
  \item pdf\_text:提取所有文字和非文字信息,包括分页符、换行符等。
  \item pdf\_data:提取数字型数据(实际使用起来效果有差异)。
  \item pdf\_fonts:提取字体信息。
  \item pdf\_attachments:提取文档附件。
  \item pdf\_toc:提取文档目录。
\end{itemize}

这几个函数所用参数完全相同,只有三个:

\begin{itemize}
  \item pdf:PDF文件路径,可以是网络链接。
  \item opw:PDF文件所有者的密码。
  \item upw:PDF文件用户的密码。
\end{itemize}

用pdf\_text提取文档时,全部内容被提取为一个字符串向量,每页内容单独放置于一个字符串中。

% 1.7节
\subsection{jsonlite——JSON文件}

JavaScript object Notation(JSON)通常作为不同语言之间互相交流信息的文件。jsonlite包能够将JSON格式文件完整解析读取到R语言中,也可以将R对象(object)输出为JSON格式。

读取JSON文件使用的是\textcolor{magenta}{fromJSON}函数。

% 1.8节
\subsection{本章小结}

不同格式数据文件读取所需要的R包:

\begin{itemize}
  \item readr/utils:平面文件格式,如.csv/txt
  \item readxl:Excel格式,如.xlsx/xls
  \item DBI:数据库格式,如.db
  \item pdftools:PDF格式
  \item jsonlite:JSON格式
\end{itemize}


% 第二章
\newpage
\section{数据清理工具}

原始数据基本都是脏的,无论人工还是传感器采集。原因很多,普遍例如:

\begin{itemize}
  \item 不同采样人员的记录数据方式不同
  \item 录入数据时的失误
  \item 传感器断电导致大段数据默认
  \item 不同国家地区对时间日期制式的标准不同
\end{itemize}

在整个数据分析过程中,数据清洗可能占到80\%的时间。耗时是因为数据清洗并非一次性工作,数据清洗→计算→可视化是一个动态循环。根据分析需求不同,清理的思路和方式也不同。例如对默认值可能采取完全移除、部分移除或替换等方式。

本章需要掌握:

\begin{itemize}
  \item 数据的“\textcolor{magenta}{脏}”和“\textcolor{magenta}{干净}”的标准是什么?
  \item 数据清理的指导原则
  \item 数据清理的相关包(\textcolor{magenta}{tibble、tidyr、lubridate、stringr}等)
\end{itemize}

% 2.1节
\subsection{基本概念}

只要不满足当前分析要求的数据集都可以说是“脏”的。而对“干净”数据可以总结为:

\begin{itemize}
  \item 属性相同的变量自成一列
  \item 单一观测自成一行
  \item 每个数据值必须独立存在
\end{itemize}

在R环境中“长”\footnote{同类型变量单独成列}数据比“宽\footnote{多个同类或者不同类的变量并存}”数据计算速度更快\footnote{两者区别文字说不清,网上找一下相关列表一眼就能看懂},两者转换只需要\textcolor{magenta}{gather}函数(见2.3节)

另外不同来源的数据应该单独成表独立存在。例如元数据\footnote{meta data,解释变量名称或数据背景的数据。通常包含坐标、指标的具体含义等解释性信息}\href{http://www.ruanyifeng.com/blog/2007/03/metadata.html}{(此处有介绍)}不应当与原始数据本身同时存在在一个数据集中。只有需要解释原始数据的时候才调用元数据。

% 2.2节
\subsection{tibble包——数据集准备}

% 2.2.1节
\subsubsection{为什么要用tibble?}

tibble包是用R中一个数据存储格式命名的。read.csv会将数据读取在一个data.frame(数据框)当中;但read\_csv有三种数据存储格式:tbl\_df、tbl和data.frame。那么为什么要用tbl格式?原因如下:

\begin{itemize}
  \item 稳定性好,可以完整保存变量名称和属性。
  \item 更多的信息展示、警示提醒,有利于及时发现错误。
  \item 新输出方式,在浏览数据时屏幕利用效率更佳。
\end{itemize}

当然tbl格式也有缺陷,就是data.frame格式用了很久,一些早期函数调用新的tbl格式可能不兼容。但是因为data.frame在处理变量名称的时候,有时会悄悄改动名称来满足自己的需求,导致会出现难以察觉的意料之外的错误,非常要命。

另外查看data.frame中变量类型往往需要str函数,但tbl无须调用函数,直接输入数据集名称即可。tbl会根据console窗口大小自动调整展示内容;而data.frame直接在console中打出来,会将小于1000行1000列的所有内容展示出来,且还不会展示变量属性等内容。孰优孰劣显而易见。

% 2.2.2节
\subsubsection{创建tbl格式}

用\textcolor{magenta}{tibble}或\textcolor{magenta}{tribble}函数创建新数据框,创建方法与data.frame函数一致。代码如下\footnote{int代表integer整数,dbl代表double浮点型数据类型}:

\begin{lstlisting}
  library(tibble)
  tibble(a = 1:6, b = a*2)
\end{lstlisting}

而tribble函数适合创建小型数据集(如元数据)。手动输入数据,变量名称用“$\sim$”开头,“,”结束,数据值用逗号分隔。元数据有时比较杂乱,用软件清理费时费力,但肉眼能直接看出关键信息时,直接用tribble函数手动生成会更高效。具体例子可以自查。

% 2.2.3节
\subsubsection{as\_tibble——转换已有格式的数据集}

\begin{itemize}
  \item is\_tibble:测试目标对象是否已经是tbl格式。
  \item \textcolor{magenta}{as\_tibble}:将vector、matrix、list\footnote{列表格式转换的时候要注意列表中要素的长度,这个具体见书P.36}和data.frame转换成tbl。
  \item enframe:也是tibble中的格式转换函数,优势在于可对向量数据进行编号。
\end{itemize}

% 2.2.4节
\subsubsection{add\_row/column}

\textcolor{magenta}{add\_row/column}函数类似Excel中任意插入或删除一行/列数据。

baseR\footnote{baseR指的应该是R语言自带的基础功能,不加任何包}也可以新增行或列:

\begin{lstlisting}
  f <- tibble(i = 1:3, j = c("John","Sam","Joy"))   # 创建一个tbl。
  f$k <- 3:1                                        # 用$来新增一列名为k的变量,数值为3、2、1。
  f[nrow(f)+1, ] <- c(4,"Jon",0)                    # nrow是计算行数,左边即为增加一行的意思。
\end{lstlisting}

上面;f[nrow(f)+1, ]中,中括号紧跟在数据框后面,可以作为索引来选择数据框中的特定数值。逗号前后分别为行、列索引\footnote{[行,列]}。因此与之相似的,f[ ,ncol(f)+1]就是新增一列的意思。

这tibble包中两个小函数实用之处在于,可以随时随地任意新增行或列数据到指定位置,不用像上面baseR中只能在数据尾部或已有变量后面新增行或列。tibble包中\textcolor{magenta}{add\_row}和\textcolor{magenta}{add\_column}使用例:

\begin{lstlisting}
  add_row(f, i = 4, j = "Jon")                      # 因为只指定了两个变量的值,所以未指定部分会自动填入默认值NA。
  add_row(f, i = 4, j = "Jon", .before = 3)         # 在第三行之前插入一行新数据。
  add_row(f, i = 4, j = "Jon", .after = 1)          # 在第一行之后插入新数据。
  add_column(f, l = nrow(f):1, .after = 1)          # 在第一列之后插入新变量,注意这里l的意思是从f的行数到1,即4到1。想了想确实应该这么写。
\end{lstlisting}

% 2.3节
\subsection{tidyr——数据清道夫}

% 2.3.1节
\subsubsection{为什么用tidyr?}

优势:

\begin{itemize}
  \item 易上手:函数名称简洁直观,可读性强
  \item 易使用:默认设置可以满足大部分需求
  \item 易记忆:不同函数的参数结构清晰
  \item 不易出现未知错误:处理数据过程中完整保留了变量属性和数据格式
\end{itemize}

% 2.3.2节
\subsubsection{gather/spread——长、宽数据转换}

\lstinline{什么是长数据?}全部变量名为一列,相关数值为一列。\textcolor{magenta}{gather}函数就是将宽数据转换成长数据。具体流程是用管道函数“\%>\%\footnote{forward-pipe operator,需要magrittr包,可以通过“?'\%>\%'”来查看具体用法。}”\href{https://www.jianshu.com/p/c65dbce983dd}{(此处有介绍)}将脏数据框df传递给gather函数。管道函数可以使得无须重新引用df,直接用“.”代替。指定指标列为key,数值列为value,保留列序号\footnote{负号+列名进行设置},并移除默认值。这会得到一个中间产物数据框,但仍然不干净,因为性别和key在同一列中。要用管道函数再次拆分,将key拆分为两列(性别+key),传递给函数separate。具体如下:

\begin{lstlisting}
  df %>%
  gather(data = ., key = key, value = value, ... = -序号, na.rm = T) %>%
  separate(data = ., key, into = c("性别", "key"))
\end{lstlisting}

因为tidy系列函数结构简洁清晰,所以熟悉之后可以省略参数名,记住位置即可:

\begin{lstlisting}
  df %>%
  gather(key, value, -序号, na.rm = T) %>%
  separate(key, c("性别", "key"))
\end{lstlisting}

如果要反过来,将长数据变换成宽数据,就用\textcolor{magenta}{spread}函数。具体操作类似gather,略。

% 2.3.3节
\subsubsection{separate/unite——拆分合并列}

\textcolor{magenta}{separate}函数类似Excel中的拆分列,该函数无法对一个单独的数值位置进行操作。\textcolor{magenta}{unite}是separate的逆向函数。

% 2.3.4节
\subsubsection{replace\_na/drop\_na——默认值处理工具}

\textcolor{magenta}{replace\_na}和\textcolor{magenta}{drop\_na}可以通过对指定列的查询来将NA替换成需要的数值。前者是替换,后者是去除默认值。使用例:

\begin{lstlisting}
  df %>%
  gather(key, value, -序号) %>%
  separate(key, c("性别", "key")) %>%
  replace_na(list(value = "missing"))
\end{lstlisting}

\begin{lstlisting}
  df %>%
  gather(key, value, -序号) %>%
  separate(key, c("性别", "key")) %>%
  drop_na()  
\end{lstlisting}

\textcolor{red}{提醒}:将所有默认值NA替换为0是很危险的,不推荐这么做。因为0代表数据存在而数值为0,但默认值NA代表数据存在或不存在两种情况,只是因为某些原因导致数据采集失败。

% 2.3.5节
\subsubsection{fill/complete——填充工具}

处理日期或者计算累计值的时候,如果中间有默认值则意味着不完整或没法计算。\textcolor{magenta}{fill}函数就类似Excel的填充功能,自动填补默认的日期或等值。而complete函数不常用就不多介绍,主要是将变量和因子的各种组合全部罗列出来。

% 2.3.6节
\subsubsection{separate\_rows/nest/unnest——行数据处理}

\begin{itemize}
  \item \textcolor{magenta}{separate\_rows}:类似Excel的拆分单元格。一个数据单位中出现多个数值,就用该函数参照参数sep给出的参进行拆分,然后顺序放入同一列的不同行中,会自动增加行数。
  \item \textcolor{magenta}{nest/unnest}:压缩和解压缩行数据。具体来说是将一个数据框压缩成新的数据框,其中包含列表型数据\footnote{这个找具体例子看一下就懂了,一目了然。}。
\end{itemize}

单独使用nest函数没有什么意义,但是配合循环或者purrr包的map函数家族时,功能性很强大。这个具体到后面看。另外注意unnest解压缩两列以上时,每一列中数据框的行数必须相等,否则无法成功解压。

% 2.4节
\subsection{lubridate日期时间处理}

% 2.4.1节
\subsubsection{为什么用lubridate?}

传感器记录的数据为了避免闰年导致奇怪的错误,一般都是纯数字的日期格式,其问题是可读性较差。对日期的处理并不是如看似那么简单,而是和处理默认值相当的一大挑战。\textcolor{magenta}{lubridate}包的优势在于:

\begin{itemize}
  \item 自然语言书写编程语法,易于理解和记忆。
  \item 融合了其他R包的时间处理函数,优化了默认配置。
  \item 能轻松完成时间日期数据的计算任务。
\end{itemize}

% 2.4.2节
\subsubsection{ymd/ymd\_hms——年月日还是日月年?}

\textcolor{magenta}{ymd及其子函数}可以完整解析数字或字符串形式的日期格式,但也有例外\footnote{例如日期中对不同成分以类似双引号作为分隔符,或是对象为奇数等。}。\textcolor{magenta}{ymd\_hms}代表年月日时分秒。lubridate函数可以用默认设置解析偶数位的字符型向量,但偶数位必须大于6位,否则会产生NA\footnote{如果函数没有解析正确的时区,用Sys.timezone()或OlsonNames()来寻找并传参正确的时区。}。

% 2.4.3节
\subsubsection{year/month/week/day/hour/minute/second——时间单位提取}

这些函数只能提取时间日期格式的对象,如常见的Date、POSIXct、POSIXlt、Period、chron、yearmon、yearqtr、zoo、zooreg等。

% 2.4.4节
\subsubsection{guess\_formats/parse\_date\_time——时间日期格式分析}

baseR不适合解析英文月份简写的日期(如24\quad Jan\quad 2018这样的),但\textcolor{magenta}{guess\_formats}和\textcolor{magenta}{parse\_date\_time}能够方便解析,思路如下:

\begin{itemize}
  \item 用guess\_formats函数猜测解析对象的可能日期顺序及格式,需要先自己指定。
  \item 复制guess\_formats函数的返回结果。
  \item 执行parse\_date\_time,将复制的内容以字符串向量格式传参给函数。
  \item 如果解析不彻底,需要手动组建日期时间格式加入到guess\_formats中的order参数。
\end{itemize}

使用例:

\begin{lstlisting}
  example_messyDate <- c("24 Jan 2018", 1802201810)
  guess_formats(example_messyDatem c("mdY", "BdY", "Bdy", "bdY", "bdy", "dbY", "dmYH"))
  parse_date_time(example_messyDate, orders = c("dObY", "dOmYH", "dmYH"))
\end{lstlisting}

% 2.5节
\subsection{stringr字符处理函数}

stringr的优点在于参数很少超过三个,参数在结构和名称上一致,逻辑清晰。简单的字符处理能极大地提高数据清理的效率。使用stingr包能:

\begin{itemize}
  \item 快速上手正则表达式——快速处理数据
  \item 理解表达式的基本概念——为以后的复杂任务打基础
\end{itemize}

% 2.5.1节
\subsubsection{baseR vs stringr}

baseR中存在一些使用正则表达式处理字符串的函数,如grep母函数类(包括了最常用的gsub,这些在Linux中也常见)。stringr简单上手但是实际处理数据时比baseR慢。因此可以通过P.52的对照表,用stringr包中的函数练习字符处理能力,实际工作中用baseR中的函数来执行。使用例:

\begin{lstlisting}
  library(stringr)
  example_txt <- "sub and gsub perform replacement of the first and all matches respectively."  # 创建练习用的字符串向量
  str_repalce(string = example_txt, pattern = "a", replacement = "@") # 参数pattern是查询第一次出现的位置,参数replacement设置为替换
  str_replace_all(string = example_txt, pattern = "a", replacement = "@") # 与上一行类似,只不过会将所有符合要求的部分全部替换掉
\end{lstlisting}

baseR中的sub和gsub函数逻辑与str\_replace和str\_replace\_all相同。

% 2.5.2节
\subsubsection{正则表达式基础}

\end{document}